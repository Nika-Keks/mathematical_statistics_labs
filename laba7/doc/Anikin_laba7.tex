\documentclass[12pt,a4paper]{article}
\usepackage[utf8]{inputenc}
\usepackage{amsmath}
\usepackage{amsfonts}
\usepackage{amssymb}
\usepackage{placeins}
\usepackage{cmap} % для кодировки шрифтов в pdf
\usepackage[T1]{fontenc}
\usepackage{hhline}
\usepackage[unicode]{hyperref}
\usepackage{multirow}
\usepackage{array}
\usepackage{amsmath}
\usepackage{bm}
\usepackage{textcomp}
\usepackage[russian]{babel}
\usepackage{graphicx} % для вставки картинок
\usepackage{amssymb,amsfonts,amsmath,amsthm} % математические дополнения от АМС
\usepackage{indentfirst} % отделять первую строку раздела абзацным отступом тоже
% Поля
\usepackage{geometry}
\geometry{left=2cm}
\geometry{right=1.5cm}
\geometry{top=2.4cm}
\geometry{bottom=2.cm}

%%%%%%%%%%%%%%%%%%%%%%%%%%%%%%%     

\linespread{1.5} % полуторный интервал
\frenchspacing

\begin{document}
	
	\begin{titlepage}
		
		\begin{center}
			\begin{large}
				Санкт-Петербургский Политехнический университет\\ Петра Великого\\
				Институт прикладной математики и механики\\
			\end{large}
			\vspace{0.2cm}
			Высшая школа прикладной математики и вычислительной физики\\
			
		\end{center}
		
		\vspace{3cm}
		\begin{center}
			\textbf{Отчёт\\ по лабораторной работе 7\\ по дисциплине\\ "математическая статистика"}
		\end{center}
		
		\vspace{3cm}
		\vbox{%
			\hfill%
			\vbox{%0
				\hbox{Выполнил студент:}%
				\hbox{\break}
				\hbox{Аникин Александр Алексеевич,}%
				\hbox{группа 3630102$\backslash$80201}%
				\hbox{\break}
				\hbox{\break}
				\hbox{Проверил:}
				\hbox{\break}
				\hbox{к.ф.-м.н., доцент}
				\hbox{Баженов Александр Николаевич}
			}%
		} 
		\vfill
		
		\begin{center}
			Санкт-Петербург\\2021
		\end{center}
		
	\end{titlepage}
	\tableofcontents
	\newpage
	
	\listoftables
	\newpage	
	
	\section{Постановка задачи}
		Сгенерировать выборку объёмом 100 элементов для нормального распределения $N(x, 0, 1)$. По сгенерированной выборке оценить параметры $\mu$ и $\sigma$ нормального закона методом максимального правдоподобия.
		В качестве основной гипотезы $H_0$ будем считать, что сгенерированное
		распределение имеет вид $N(x, \hat{\mu},\hat{\sigma})$. Проверить основную гипотезу, используя критерий согласия $\chi^2$. В качестве уровня значимости взять $\alpha = 0.05$. Привести таблицу вычислений $\chi^2$.
		Исследовать точность (чувствительность) критерия - сгенерировать выборки равномерного распределения и распределения Лапласа малого объема, проверить их на нормальность.
		\newpage
	
	\section{Теория}
		\subsection{Метод максимального правдоподобия}
			$L(x_1, ... , x_n, \theta)$ - функция правдоподобия, рассматриваемая как функция неизвестного параметра $\theta$:
			\begin{equation}
				L(x_1, ... , x_n, \theta)=f(x_1, \theta)f(x-2, \theta)...f(x_n, \theta)	
			\end{equation}
			Оценка максимального правдоподобия:
			\begin{equation}
				\hat{\theta} = \arg{\underset{\theta}\max} L(x_1, ... , x_n, \theta)
			\end{equation}
			Система уравнений правдоподобия (в случае дифференцируемости функции правдоподобия):
			\begin{equation}
				\frac{\partial{l}}{\partial{\theta_k}}=0 \quad \text{или} \quad \frac{\partial{\ln{l}}}{\partial{\theta_k}}=0, \quad k=1,...,m
			\end{equation}
	
		\subsection{Проверка гипотезы о законе распределения генеральной совокупности критерием $\chi^2$}
			Выдвинута гипотеза $H_0$ о генеральном законе распределения с функцией
			распределения $F(x)$.
			Рассматриваем случай, когда гипотетическая функция распределения $F(x)$
			не содержит неизвестных параметров.\\
			\textbf{Правило проверки гипотезы о законе распределения критерием $\chi^2$:}
			\begin{itemize}
				\item Выбирается уровень значимости $\alpha$;
				\item По таблице (\cite{maksimov}, стр. 358) выбирается квантиль $\chi_{1-\alpha}^2(k-1)$ порядка 1-$\alpha$ с $k-1$ степенями свободы;
				\item С помощью гипотетической функции распределения $F(x)$ вычисляются вероятности $p_i = P(X \in \Delta_i), \quad i=1,...,k$;
				\item Находятся частоты $n_i$ попадания элементов выборки в поднмножества $\Delta_i, i=1,..$ 
				\item Вычисляется выборочное значение статистики критерия $\chi^2$:
				\begin{equation}
					\chi_B^2=\sum_{i=1}^{k}\frac{(n_i-np_i)^2}{np_i}
				\end{equation} 
				\item Сравниваются $\chi_B^2$ и квантиль $\chi_{1-\alpha}^2(k-1)$:
				\begin{itemize}
					\item если $\chi_B^2 < \chi_{1-\alpha}^2(k-1)$, то гипотеза $H_0$ на данном этапе проверки принимается;
					\item если $\chi_B^2  \chi_{1-\alpha}^2(k-1)$, то гипотеза $H_0$ отвергается, выбирается
					одно из альтернативных распределений, и процедура проверки повторяется.
				\end{itemize}
			\end{itemize}
		\newpage
		
		\section{Реализация}
		Лабораторная работа выполнена на языке Python 3.8 с помощью загружаемых пакетов SciPy, NumPy. Исходный код лабораторной работы находится на GitHub репозитории.
		\newpage
		
		\section{Результаты}
			\subsection{Проверка гипотезы о законе распределения генеральной совокупности критерием $\chi^2$}
				\subsubsection{Нормальное распределение $N(x, \hat{\mu}, \hat{\sigma})$}
				Основная гипотеза $H_0$: $F(x) = N(x, \bar{\mu}, \bar{\sigma})$.\\
				Метод максимального правдоподобия:
				\begin{center}
					$\hat{\mu}=-0.035 \quad \hat{\sigma} = 1.041$
				\end{center}
				Критерий $\chi^2:$
				\begin{itemize}
					\item Количество промежутков: $k=7$;
					\item Уровень значимости: $\alpha=0.05$;
					\item Квантиль распределения $\chi^2_{0.95}(6)=12.59$;
				\end{itemize}
				\begin{table}[h!]
					\begin{center}
						\begin{tabular}{|c|c|c|c|c|c|c|}
							\hline
							i & границы $\Delta_i$ & $n_i$ & $p_i$ & $np_i$ & $n_i-np_i$ & $\chi_i^2$ \\ \hline
							1 & [ -$\infty$ , -1.0 ] & 18 & 0.175 & 17.530 & 0.47 & 0.013 \\ \hline
							2 & [ -1.0 , -0.6 ] & 11 & 0.114 & 11.374 & -0.374 & 0.0120 \\ \hline
							3 & [ -0.6 , -0.2 ] & 12 & 0.140 & 13.994 & -1.994 & 0.284 \\ \hline
							4 & [ -0.2 , 0.2 ] & 16 & 0.150 & 14.960 & 1.040 & 0.072 \\ \hline
							5 & [ 0.2 , 0.6 ] & 17 & 0.139 & 13.894 & 3.106 & 0.694 \\ \hline
							6 & [ 0.6 , 1.0 ] & 7 & 0.112 & 11.211 & -4.211 & 1.582 \\ \hline
							7 & [ 1.0 , $\infty$ ] & 19 & 0.170 & 17.037 & 1.963 & 0.226 \\ \hline
							$\sum$ & - & 100 & 1 & 100 & 0 & $\chi_B^2=2.146$ \\ \hline
						\end{tabular}
					\caption{Нормальное распределение, проверка гипотезы}
					\end{center}
				\end{table}
				$\chi_B^2 = 2.146 <  \chi^2_{0.95}(6)=12.59$, значит, на данном этапе проверки текущая гипотеза принимается.
				
				\newpage
				\subsubsection{Исследование критерия $\chi^2$ на чувствительность}
				Генерируются выборки равномерного распределения и распределения Лапласа по 20 элементов и проверяется гипотеза, что полученные наборы данных являются выборками нормального распределения.
				\begin{itemize}
					\item Равномерное распределение $U(x, -1.5, 1.5)$:\\
					Метод максимального правдоподобия:
					\begin{center}
						$\hat{\mu}=0.421 \quad \hat{\sigma} = 0.577$
					\end{center}
					Критерий $\chi^2:$
					\begin{itemize}
						\item Количество промежутков: $k=5$;
						\item Уровень значимости: $\alpha=0.05$;
						\item Квантиль распределения $\chi^2_{0.95}(4)=9.49$;
					\end{itemize}
					\begin{table}[h!]
						\begin{center}
							\begin{tabular}{|c|c|c|c|c|c|c|}
								\hline
								i & границы $\Delta_i$ & $n_i$ & $p_i$ & $np_i$ & $n_i-np_i$ & $\chi_i^2$ \\ \hline
								1 & [ $-\infty$ , -1.0 ] & 1.0 & 0.159 & 3.173 & -2.173 & 1.488 \\ \hline
								2 & [ -1.0 , -0.333 ] & 3.0 & 0.211 & 4.216 & -1.216 & 0.351 \\ \hline
								3 & [ -0.333 , 0.333 ] & 6.0 & 0.261 & 5.222 & 0.778 & 0.116 \\ \hline
								4 & [ 0.333 , 1.0 ] & 4.0 & 0.211 & 4.216 & -0.216 & 0.011 \\ \hline
								5 & [ 1.0 , $\infty$ ] & 6.0 & 0.159 & 3.173 & 2.827 & 2.518 \\ \hline
								$\sum$ & - & 20 & 1 & 20 & 0 & $\chi_B^2=4.484$ \\ \hline
							\end{tabular}
						\caption{Равномерное распределение, проверка на устойчивость}
						\end{center}
					\end{table}
					$\chi_B^2 = 4.484 <  \chi^2_{0.95}(6)=9.49$, значит, на данном этапе проверки текущая гипотеза принимается.
					
					\newpage
					\item Распределение Лапласа $L(x, 0, 1)$:\\
					Метод максимального правдоподобия:
					\begin{center}
						$\hat{\mu}=0.165 \quad \hat{\sigma} = 1.232$
					\end{center}
					Критерий $\chi^2:$
					\begin{itemize}
						\item Количество промежутков: $k=5$;
						\item Уровень значимости: $\alpha=0.05$;
						\item Квантиль распределения $\chi^2_{0.95}(4)=9.49$;
					\end{itemize}
				
					\FloatBarrier
					\begin{table}[h!]
						\begin{center}
							\begin{tabular}{|c|c|c|c|c|c|c|}
								\hline
								i & границы $\Delta_i$ & $n_i$ & $p_i$ & $np_i$ & $n_i-np_i$ & $\chi_i^2$ \\ \hline
								1 & [ $-\infty$ , -1.0 ] & 2.0 & 0.159 & 3.173 & -1.173 & 0.434 \\ \hline
								2 & [ -1.0 , -0.333 ] & 7.0 & 0.211 & 4.216 & 2.784 & 1.839 \\ \hline
								3 & [ -0.333 , 0.333 ] & 4.0 & 0.261 & 5.222 & -1.222 & 0.286 \\ \hline
								4 & [ 0.333 , 1.0 ] & 3.0 & 0.211 & 4.216 & -1.216 & 0.351 \\ \hline
								5 & [ 1.0 , $\infty$ ] & 4.0 & 0.159 & 3.173 & 0.827 & 0.215 \\ \hline
								$\sum$ & - & 20 & 1 & 20 & 0 & $\chi_B^2=3.124$ \\ \hline
							\end{tabular}
						\caption{Распределение Лапласа, проверка на устойчивость}
						\end{center}
					\end{table}
					\FloatBarrier
					
					$\chi_B^2 = 3.124 <  \chi^2_{0.95}(6)=9.49$, значит, на данном этапе проверки текущая гипотеза принимается.
				\end{itemize}
		\newpage
		\clearpage
	\section{Обсуждение}
	Проведенное исследование показало, что метод $\chi^2$ неэффективен для выборок малого размера - по результатам исследования на чувствительность выборки распределения Лапласа и равномерного распределения воспринимались как выборки нормального распределения, поэтому для более точной проверки гипотез о законах распределения следует проводить проверку на большем объеме данных.
	\newpage
			
	\begin{thebibliography}{1}
		\addcontentsline{toc}{section}{\bibname}
		\bibitem{maksimov}  Максимов Ю.Д. Математика. Теория и практика по математической статистике. Конспект-справочник по теории вероятностей : учеб. пособие /
		Ю.Д. Максимов; под ред. В.И. Антонова. — СПб. : Изд-во Политехн.
		ун-та, 2009. — 395 с. (Математика в политехническом университете).

	\end{thebibliography}
\end{document}