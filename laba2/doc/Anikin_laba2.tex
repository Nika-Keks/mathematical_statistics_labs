\documentclass[12pt,a4paper]{article}
\usepackage[utf8]{inputenc}
\usepackage{amsmath}
\usepackage{amsfonts}
\usepackage{amssymb}

\usepackage{cmap} % для кодировки шрифтов в pdf
\usepackage[T1]{fontenc}
\usepackage[unicode]{hyperref}
\usepackage{hhline}
\usepackage{multirow}
\usepackage{array}
\usepackage{amsmath}
\usepackage{bm}
\usepackage{textcomp}
\usepackage[russian]{babel}
\usepackage{graphicx} % для вставки картинок
\usepackage{amssymb,amsfonts,amsmath,amsthm} % математические дополнения от АМС
\usepackage{indentfirst} % отделять первую строку раздела абзацным отступом тоже
% Поля
\usepackage{geometry}
\geometry{left=2cm}
\geometry{right=1.5cm}
\geometry{top=2.4cm}
\geometry{bottom=2.cm}

%%%%%%%%%%%%%%%%%%%%%%%%%%%%%%%     

\linespread{1.5} % полуторный интервал
\frenchspacing

\begin{document}
	
	\begin{titlepage}
		
		\begin{center}
			\begin{large}
				Санкт-Петербургский Политехнический университет\\ Петра Великого\\
				Институт прикладной математики и механики\\
			\end{large}
			\vspace{0.2cm}
			Высшая школа прикладной математики и вычислительной физики\\
			
		\end{center}
		
		\vspace{3cm}
		\begin{center}
			\textbf{Отчёт\\ по лабораторной работе №2\\ по дисциплине\\ "математическая статистика"}
		\end{center}
		
		\vspace{3cm}
		
		\vbox{%
			\hfill%
			\vbox{%
				\hbox{Выполнил студент:}%
				\hbox{\break}
				\hbox{Аникин Александр Алексеевич,}%
				\hbox{группа 3630102$\backslash$80201}%
				\hbox{\break}
				\hbox{\break}
				\hbox{Проверил:}
				\hbox{\break}
				\hbox{к.ф.-м.н., доцент}
				\hbox{Баженов Александр Николаевич}
			}%
		} 
		\vfill
		
		\begin{center}
			Санкт-Петербург, 2021
		\end{center}
		
	\end{titlepage}
	\tableofcontents
	\newpage
	
	\listoftables
	\newpage
	
	\section{Постановка задачи}
	Для следующих распределений:
	\begin{itemize}
		\item Нормальное распределение $\textit{N}(\textit{x}, 0, 1)$
		\item Распределение Коши $\textit{C}(\textit{x}, 0, 1)$
		\item Распределение Лапласа $\textit{L}(\textit{x}, 0, \frac{1}{\sqrt{2}})$
		\item Распределение Пуассона $\textit{P}(\textit{k}, 10)$
		\item Равномерное распределение $\textit{U}(\textit{x}, -\sqrt{3}, \sqrt{3})$
	\end{itemize}
	Сгенерировать выборки размером 10, 100 и 1000 элементов.
	Для каждой выборки вычислить следующие статистические характеристики положения данных: $\overline{\rm x}, med \ x, z_R, z_Q, z_{tr}$. Повторить такие
	вычисления 1000 раз для каждой выборки и найти среднее характеристик положения и их квадратов:
	\begin{equation}\label{E(z)}
		E(z)=z
	\end{equation}
	Вычислить оценку дисперсии по формуле:
	\begin{equation}
		D(z)=\overline{\rm z^2}-\overline{z}^2\label{D(Z)}
	\end{equation}
	Представить полученные данные в виде таблиц.
	
	\newpage
	
	\section{Теория}
		\subsection{Рассматриваемые распределения}
		Плотности:
			\begin{itemize}
				\item Нормальное распределение:
				\begin{equation}\label{norm}
					\textit{N}(\textit{x}, 0, 1)=\frac{1}{\sqrt{2\pi}}e^{-\frac{x^2}{2}}
				\end{equation}
				
				\item Распределение Коши:
				\begin{equation}\label{cauchy}
					\textit{C}(\textit{x}, 0, 1)=\frac{1}{\pi}\frac{1}{x^2+1}
				\end{equation}
				
				\item Распределение Лапласа:
				\begin{equation}\label{laplace}
					\textit{L}(\textit{x}, 0, \frac{1}{\sqrt{2}})=\frac{1}{\sqrt{2}}e^{-\sqrt{2}|x|}
				\end{equation}
				
				\item Распределение Пуассона:
				\begin{equation}\label{poisson}
					\textit{P}(\textit{k}, 10)=\frac{10^k}{k!}e^{-10}
				\end{equation}
				
				\item Равномерное распределение:
				\begin{equation}\label{uniform}
					\textit{U}(\textit{x}, -\sqrt{3}, \sqrt{3})=
					\left\{
					\begin{array}{l}
						\frac{1}{2\sqrt{3}} \quad \text{при} \quad |x|\leq \sqrt{3}\\
						0 \quad \quad \text{при} \quad |x|>3
					\end{array}
					\right.
				\end{equation}
			\end{itemize}
		 	
	 
		\subsection{Вариационный ряд}
			Вариационный ряд - последовательность элементов выборки, расположенных в неубывающем порядке. Одинаковые элементы повторяются \cite{num_chars}.
			
	 	\subsection{Выборочные характеристики}
	 		\subsubsection{Характеристики положения}
	 			\begin{itemize}
	 				\item Выборочное среднее:
	 				\begin{equation}\label{mean}
	 					\overline{x}=\frac{1}{n}\sum_{i=1}^{n}x_i
	 				\end{equation}
 				
 					\item Выборочная медиана:
 					\begin{equation}\label{median}
 						med x=
 						\left\{
	 						\begin{array}{l}
								x_{l+1} \qquad \text{при} \quad n=2l+1 \\
	 							\frac{x_l+x_{l+1}}{2} \quad \text{при} \quad n=2l
 							\end{array}
 						\right.
 					\end{equation}
 				
 					\item Полусумма экстремальных выборочных элементов:
 					\begin{equation}\label{half_sum_ext}
 						z_R=\frac{x_1+x_n}{2}
 					\end{equation}
 				
 					\item Полусумма квартилей:
 						\subitem Выборочная квартиль $z_p$ порядка $p$ определяется формулой:
		 					\begin{equation}\label{half_sum_quart}
								z_p=
								\left\{
								\begin{array}{l}
									x_{np+1} \quad \text{при } np \text{ дробном} \\
									x_{np} \qquad \text{при } np \text{ целом}
								\end{array}
								\right.
		 					\end{equation}
 						\subitem Полусумма квартилей:
 							\begin{equation}
 								z_Q=\frac{z_{\frac{1}{4}}+z_{\frac{3}{4}}}{2}
	 						\end{equation}
 					
 					\item Усечённое среднее:
 						\begin{equation}\label{trunc_mean}
 							z_{tr}=\frac{1}{n-2r}\sum_{i=r+1}^{n-r}x_i,\quad r\approx\frac{n}{4}
 						\end{equation}
	 			\end{itemize}
			\subsubsection{Характеристики рассеяния}
				\begin{itemize}
					\item Выборочная дисперсия:
					\begin{equation}
						D=\frac{1}{n}\sum_{i=1}^{n}(x_i-\overline{x})^2
					\end{equation}
				\end{itemize}
	\newpage
	
	\section{Реализация}
	Лабораторная работа выполнена на языке Python 3.8 с помощью загружаемого пакета SciPy. Исходный код лабораторной работы находится на GitHub репозитории.
	\newpage
	
	\section{Результаты}
		\subsection{Характеристики положения и рассеяния}
		Интервальная оценка $E$ рассчитывалась следующим образом:\\
		\begin{equation}\label{E}
			E=E(z)\pm{\sqrt{D(z)}}
		\end{equation}
			\begin{table}[htp]
				\label{tabular:normal}
				\begin{center}
					\begin{tabular}{|c|c|c|c|c|c|} 
						
						\hline
						normal n=10 & \ & \ & \ & \ & \ \\ \hline
						\ & $\overline{x}$ (\ref{mean}) & $med \ x$ (\ref{median}) & $z_R$ (\ref{half_sum_ext}) & $z_Q$ (\ref{half_sum_quart}) & $z_tr$ (\ref{trunc_mean})\\ \hline
						E(z) (\ref{E(z)}) & 0.0011 & 0.2488 & 0.0064 & 0.2206 & 0.1082 \\ \hline
						D(z) (\ref{D(Z)}) & 0.0923   & 0.1326 & 0.1794 &   0.1089   &  0.0750 \\ \hline
						E (\ref{E})& 0 & 0 & 0 & 0 & 0\\\hline
						\ & \ & \ & \ & \ & \ \\ \hline
						
						normal n=100 & \ & \ & \ & \ & \ \\ \hline
						\ & $\overline{x}$ & $med \ x$  & $z_R$  & $z_Q$  & $z_tr$ \\ \hline
						E(z) & 0.0009 & 0.02699 &  0.0035 & 0.0233 & 0.0136 \\ \hline
						D(z) & 0.0095   & 0.0152 & 0.0901 &  0.0124   &  0.0115 \\ \hline
						E & 0.0 & 0 & 0 & 0 & 0\\\hline
						\ & \ & \ & \ & \ & \ \\ \hline
						
						normal n=1000 & \ & \ & \ & \ & \ \\ \hline
						\ & $\overline{x}$ & $med \ x$  & $z_R$  & $z_Q$  & $z_tr$ \\ \hline
						E(z) & -0.0002 & 0.0037 &  0.0083 & 0.0022 & 0.0019 \\ \hline
						D(z) & 0.0010   & 0.0016 & 0.0651 &  0.0012   &  0.0012 \\ \hline
						E & 0.0 & 0.0 & 0 & 0.0 & 0.0\\\hline
						
					\end{tabular}
				\end{center}
			\caption{Нормальное распределение (\ref{norm}), выборочные характеристики}
			\end{table}
		
			\begin{table}[htp]
				\label{tabular:cauchy}
				\begin{center}
					\begin{tabular}{|c|c|c|c|c|c|} 
						
						\hline
						cauchy n=10 & \ & \ & \ & \ & \ \\ \hline
						\ & $\overline{x}$ & $med \ x$ & $z_R$ & $z_Q$ & $z_tr$ \\ \hline
						E(z) & -0.7997 & 0.3711 & -3.8942 & 0.7932 & 0.2012 \\ \hline
						D(z) & 1496.1612  & 0.3888 & 37210.1136 &  4.0337  &  0.3009 \\ \hline
						E & - & 0 & - & - & 0\\\hline
						\ & \ & \ & \ & \ & \ \\ \hline
						
						cauchy n=100 & \ & \ & \ & \ & \ \\ \hline
						\ & $\overline{x}$ & $med \ x$  & $z_R$  & $z_Q$  & $z_tr$ \\ \hline
						E(z) & 1.7012 & 0.0394 &  83.5465 & 0.05417 & 0.02892 \\ \hline
						D(z) & 23276.8756   & 0.0266 & 57935136.5867 &  0.0546   &  0.0274 \\ \hline
						E & - & 0 & - & 0 & 0\\\hline
						\ & \ & \ & \ & \ & \ \\ \hline
						
						cauchy n=1000 & \ & \ & \ & \ & \ \\ \hline
						\ & $\overline{x}$ & $med \ x$  & $z_R$  & $z_Q$  & $z_tr$ \\ \hline
						E(z) & -0.7303 & 0.0007 &  -322.6777 & 0.0047 & 0.0003 \\ \hline
						D(z) & 413.70   & 0.0025 & 101229229.65 &  0.0052   &  0.0027 \\ \hline
						E & - & 0.0 & - & 0.0 & 0.0\\\hline
						
					\end{tabular}
				\end{center}
				\caption{Распределение Коши (\ref{cauchy}), выборочные характеристики}
			\end{table}
			
			\begin{table}[htp]
				\label{tabular:laplace}
				\begin{center}
					\begin{tabular}{|c|c|c|c|c|c|} 
						
						\hline
						laplace n=10 & \ & \ & \ & \ & \ \\ \hline
						\ & $\overline{x}$ & $med \ x$ & $z_R$ & $z_Q$ & $z_tr$ \\ \hline
						E(z) & 0.0069 & 0.1944 & 0.0087 & 0.2110 & 0.0922 \\ \hline
						D(z) & 0.1113  & 0.088 & 0.4876 &  0.1093  &  0.0545 \\ \hline
						E & 0 & 0 & 0 & 0 & 0\\\hline
						\ & \ & \ & \ & \ & \ \\ \hline
						
						laplace n=100 & \ & \ & \ & \ & \ \\ \hline
						\ & $\overline{x}$ & $med \ x$  & $z_R$  & $z_Q$  & $z_tr$ \\ \hline
						E(z) & 0.0019 & 0.0164 &  -0.0237 & 0.0202 & 0.0097 \\ \hline
						D(z) & 0.0101   & 0.0060 & 0.4103 &  0.0095 & 0.0060 \\ \hline
						E & 0.0 & 0.0 & 0 & 0 & 0.0\\\hline
						\ & \ & \ & \ & \ & \ \\ \hline
						
						laplace n=1000 & \ & \ & \ & \ & \ \\ \hline
						\ & $\overline{x}$ & $med \ x$  & $z_R$  & $z_Q$  & $z_tr$ \\ \hline
						E(z) & -0.0013 & 0.0014 &  -0.058 & 0.0012 & 0.00026 \\ \hline
						D(z) & 0.0010   & 0.00052 & 0.45 &  0.0010   &  0.00063 \\ \hline
						E & 0.0 & 0.0 & 0 & 0.0 & 0.0\\\hline
						
					\end{tabular}
				\end{center}
				\caption{Распределение Лапласа (\ref{laplace}), выборочные характеристики}
			\end{table}
		
			\begin{table}[htp]
				\label{tabular:poisson}
				\begin{center}
					\begin{tabular}{|c|c|c|c|c|c|} 
						
						\hline
						poisson n=10 & \ & \ & \ & \ & \ \\ \hline
						\ & $\overline{x}$ & $med \ x$ & $z_R$ & $z_Q$ & $z_tr$ \\ \hline
						E(z) & 9.96 & 10.608 & 10.2545 & 10.608 & 8.542 \\ \hline
						D(z) & 1.0271  & 1.55 & 1.9251 &  1.24  &  0.86 \\ \hline
						E & $9^{+1}_{-1}$ & $10^{+1}_{-1}$ & $10^{+2}_{-1}$ & $10^{+1}_{-1}$ & $8^{+1}_{-1}$\\\hline
						\ & \ & \ & \ & \ & \ \\ \hline
						
						poisson n=100 & \ & \ & \ & \ & \ \\ \hline
						\ & $\overline{x}$ & $med \ x$  & $z_R$  & $z_Q$  & $z_tr$ \\ \hline
						E(z) & 10.00 & 9.909 &  10.882 & 9.9805 & 9.70 \\ \hline
						D(z) & 0.11   & 0.20 & 1.015 &  0.16 & 0.12 \\ \hline
						E & $10^{+0}_{-1}$ & $9^{+1}_{-0}$ & $10^{+1}_{-1}$ & $9^{+1}_{-0}$ & $9^{+1}_{-0}$\\\hline
						\ & \ & \ & \ & \ & \ \\ \hline
						
						poisson n=1000 & \ & \ & \ & \ & \ \\ \hline
						\ & $\overline{x}$ & $med \ x$  & $z_R$  & $z_Q$  & $z_tr$ \\ \hline
						E(z) & 10.0045 & 9.997 &  11.6715 & 9.99675 & 9.85 \\ \hline
						D(z) & 0.0098   & 0.0025 & 0.61 &  0.0020  &  0.011 \\ \hline
						E & $10^{+0}_{-1}$ & $9^{+1}_{-0}$ & $11^{+1}_{-1}$ & $9^{+1}_{-0}$ & $9^{+1}_{-0}$ \\\hline
						
					\end{tabular}
				\end{center}
				\caption{Распределение Пуассона (\ref{poisson}), выборочные характеристики}
			\end{table}
			
			\begin{table}[htp]
				\label{tabular:uniform}
				\begin{center}
					\begin{tabular}{|c|c|c|c|c|c|} 
						
						\hline
						uniform n=10 & \ & \ & \ & \ & \ \\ \hline
						\ & $\overline{x}$ & $med \ x$ & $z_R$ & $z_Q$ & $z_tr$ \\ \hline
						E(z) & -0.0085 & 0.3012 & -0.0020 & 0.2353 & 0.1212 \\ \hline
						D(z) & 0.0974 & 0.2141 & 0.0484 &  0.1312  &  0.1254 \\ \hline
						E & 0 & 0 & 0 & 0 & 0\\\hline
						\ & \ & \ & \ & \ & \ \\ \hline
						
						uniform n=100 & \ & \ & \ & \ & \ \\ \hline
						\ & $\overline{x}$ & $med \ x$  & $z_R$  & $z_Q$  & $z_tr$ \\ \hline
						E(z) & -0.0021 & 0.0395 &  -0.0009 & 0.0211 & 0.0157 \\ \hline
						D(z) & 0.0113   & 0.0313 & 0.0005 &  0.0154 & 0.0208 \\ \hline
						E & 0 & 0 & 0.0 & 0 & 0\\\hline
						\ & \ & \ & \ & \ & \ \\ \hline
						
						uniform n=1000 & \ & \ & \ & \ & \ \\ \hline
						\ & $\overline{x}$ & $med \ x$  & $z_R$  & $z_Q$  & $z_tr$ \\ \hline
						E(z) & -0.0006 & 0.0012 & 0.0001 & 0.0024 & 0.0008 \\ \hline
						D(z) & 0.0010   & 0.0028 & 0.0000 &  0.0015  &  0.0019 \\ \hline
						E & 0.0 & 0.0 & 0.00 & 0.0 & 0.0\\\hline
						
					\end{tabular}
				\end{center}
				\caption{Равномерное распределение (\ref{uniform}), выборочные характеристики}
			\end{table}
		\clearpage
	\newpage
	
	\begin{thebibliography}{1}
		\addcontentsline{toc}{section}{\bibname}
		\bibitem{num_chars}  Вероятностные разделы математики. Учебник для бакалавров технических направлений.//Под ред. Максимова Ю.Д. — Спб.: «Иван Федоров», 2001. — 592 c., илл.
	\end{thebibliography}
	
		
\end{document}